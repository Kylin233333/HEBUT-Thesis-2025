\pagenumbering{Roman}
\newpage
\begin{center}
    \fontsize{18}{1}\fzxbsjt{毕业设计中文摘要}
\end{center}
\begin{tcolorbox}
\begin{center}
    \fontsize{16}{0}\heiti{\textbf{生成模型研究与中国画生成}}
\end{center}
\vspace{8mm}
\noindent
{\fontsize{16}{0}\heiti{\textbf{摘要:}}}
\vspace{2mm}

\setlength{\parindent}{24pt}
生成模型是人工智能领域的重要研究方向,研究生成模型,有助于人工智能更好地理解现实世界。
近年来,生成模型在图像或文字生成方面取得了巨大突破。
在图像领域,生成模型生成的图像质量已经可以与真实图像比肩,
在文本生成方面,ChatGPT已经可以与人们自然地交谈。

生成模型是近年来非常活跃的研究领域,对生成模型进行系统回顾总结有助于对其更好地研究。
因此,本文对近年来的主要生成模型进行了较为系统地介绍,并根据是否直接定义概率密度函数,
将生成模型分为显式密度模型和隐式密度模型。
此外,对生成模型在中国画生成上的应用进行探索,使用降噪扩散模型生成中国画。
\vspace{8mm}

\noindent\textbf{关键词}: \quad 生成模型 \quad 中国画 \quad 降噪扩散模型
% \vspace{80mm}
\end{tcolorbox}


\newpage
\begin{center}
    \fontsize{18}{1}\fzxbsjt{毕业设计外文摘要}
\end{center}
\begin{tcolorbox}
\pdfbookmark[section]{ABSTRACT}{ABSTRACT}
\begin{center}
    \fontsize{14}{0}\songti{\textbf{Generative Models Research and Chinese Painting Generation}}
\end{center}
\vspace{8mm}
\noindent
{\fontsize{16}{0}\heiti{\textbf{ABSTRACT}}}
\vspace{2mm}

\setlength{\parindent}{24pt}
Generative model is an impoertant reserach direction in the field of artificial intelligence.
The research of generative model is helpful for artificial intelligence to understand the world better.
In recent years, the generative model has made great breakthroughs in image and text generation.
In the field of image, the quality of the images generated by the generative model can already be comparable to the real image.
In terms of text generation, ChatGPT can already talk to people naturally.

Generative model is a very active research field in recent years,
Systematice review and summary of generative model is helpful to better study it.
Therefore, this paper systematically introduces the main generative models in recent years, 
and depending on whether the probability density function is directly defined,
the generated models are divided into explicit density model and implicit density model.
In addition, the application of the generative model in the generation of Chinese painting is explored, 
and the diffusion model is used to generate Chinese painting.
\vspace{8mm}

\noindent\textbf{Keywords}: Generative Models, Chinese Painting, Denoising Diffusion Model
% \vspace{28mm}
\end{tcolorbox}

\setlength{\parindent}{24pt}

% 生成模型研究与中国画生成
% Generative Models Research and Chinese Painting Generation

% 基于GAN的中国山水画生成系统设计与开发
% Design and development of Chinese landscape painting generation system based on GAN