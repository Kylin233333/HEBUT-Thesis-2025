% !TeX root = ../HebutThesis_example.tex(此文件是被HebutThesis_example.tex调用的)
% 在此文件中编辑摘要页里需要填写的内容

\pagenumbering{Roman}   % 页面用罗马字体编号

% 中文摘要和关键词
\newpage
\begin{center}
    \fontsize{18}{1}\fzxbsjt{毕业设计(论文)中文摘要}
\end{center}
\begin{tcolorbox}
\begin{center}
    {\fontsize{16}{0}\heiti{\textbf{您的论文题目}}}   % 此处修替换为您的论文题目
\end{center}
\vspace{8mm}
% 中文摘要
\noindent
{\fontsize{16}{0}\heiti{\textbf{摘要:}}}
\vspace{2mm}
\setlength{\parindent}{24pt}
  % 此处替换为您的中文摘要内容,另起一段需要空一行

  本科毕业设计论文摘要是对整个研究工作的高度概括,
  通常包含研究的背景、目的、方法、主要结果和结论。
  具体写法应简洁明了,遵循以下步骤:

  1、背景:简短介绍研究领域和研究的意义。

  2、目的:明确指出研究旨在解决的问题或目标。

  3、方法:概述采用的研究方法或技术。

  4、结果:简要描述研究的主要发现或成果。

  5、结论:提出研究的主要结论,可能包括对未来研究的建议。

  本科毕业设计论文的关键词应精准概括论文的核心内容和研究领域,
  有助于读者和数据库检索系统快速理解和定位论文。写法要点如下:

  1、相关性:选择与论文主题紧密相关的词汇。

  2、专业性:使用专业领域内常用的术语或词汇。

  3、具体性:避免过于宽泛的词汇,关键词应具体反映论文的研究内容。

  4、数量:一般选取3-5个关键词,足以覆盖论文的主要研究领域和特点。

  5、排序:按照关键词的重要程度或与论文内容的相关性排序。

  关键词之间用逗号分隔,放在摘要下方,
  有时候也可以包括研究方法和地点等信息,
  以提高论文的可检索性。
  
\vspace{8mm}

% 中文关键词
\noindent
{\fontsize{12}{0}\heiti\textbf{关键词}}: \quad 关键词 1 \quad 关键词 2 \quad 关键词 3  % 此处替换为您的中文关键词
\end{tcolorbox}

% 外文摘要和关键词
\newpage
\begin{center}
    \fontsize{18}{1}\fzxbsjt{毕业设计(论文)外文摘要}
\end{center}
\begin{tcolorbox}
\pdfbookmark[section]{ABSTRACT}{ABSTRACT}
\begin{center}
    \fontsize{14}{0}\songti{\textbf{Your Thesis Title}}   % 此处替换为您的论文外文标题
\end{center}
\vspace{8mm}
% 外文摘要
\noindent
{\fontsize{16}{0}\heiti{\textbf{ABSTRACT}}}
\vspace{2mm}
\setlength{\parindent}{24pt}
  % 此处替换为您的外文摘要内容,另起一段需要空一行

  The abstract of the undergraduate graduation thesis is a high-level summary of the entire research work, usually including the background of the research
  Purpose, Method, Main Results, and Conclusion. The specific writing should be concise and clear, following the following steps:
  1. Background: Briefly introduce the research field and its significance.
  2. Purpose: Clearly indicate the problem or goal that the research aims to solve.
  3. Method: Summarize the research methods or techniques used.
  4. Result: Briefly describe the main findings or outcomes of the study.
  5. Conclusion: Present the main conclusions of the study, which may include recommendations for future research.

  The keywords for undergraduate graduation thesis should accurately summarize the core content and research field of the thesis, which is helpful for reading
  Quickly understand and locate papers through the database retrieval system. The key points of writing are as follows:
  1. Relevance: Choose vocabulary closely related to the topic of the paper.
  2. Professionalism: Using commonly used terms or vocabulary in the professional field.
  3. Specific: Avoid overly broad vocabulary, and keywords should specifically reflect the research content of the paper.
  Separate keywords with commas and place them below the abstract, sometimes including research methods and locations
  Information to improve the searchability of the paper.

\vspace{8mm}

% 外文关键词
\noindent
{\fontsize{12}{0}\heiti\textbf{Keywords}}: keyword 1, keyword 2, keyword 3    % 此处替换为您的外文关键词
% \vspace{28mm}
\end{tcolorbox}

\setlength{\parindent}{24pt}

