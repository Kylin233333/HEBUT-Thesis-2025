% !TeX root = ../HebutThesis_example.tex(此文件是被HebutThesis_example.tex调用的)
\chapter{数学符号与公式(仅供参考)}
\section{数学模式}
在LaTeX中,数学内容可以在文本内部(内联模式)或作为单独的段落(显示模式)呈现。
% 内联模式:使用$...$或\(...\)将数学内容包围起来。
% 例如:$a^2 + b^2 = c^2$。
内联公式示例:$E=mc^2$。

% 显示模式:使用$$...$$或\[...\]将数学内容包围起来,这会使公式居中显示并且与周围的文本分开。
% 例如:\[a^2 + b^2 = c^2\]。
显示公式示例:
\[E=mc^2\]

\section{基础数学符号}

希腊字母:
\[\alpha, \beta, \gamma, \Gamma\]

上标和下标:
% ^表示上标,_表示下标。
\[x_i^2\]

求和公式:
\[\sum_{n=1}^{\infty} \frac{1}{n^2} = \frac{\pi^2}{6}\]

分数:
% \frac{分子}{分母}。
\[\frac{x}{y}\]

根号:
% \sqrt{表达式}表示平方根,\sqrt[n]{表达式}表示n次根。

\[\sqrt{x+y}\]
\[\sqrt[3]{x}\]

求和与积分:
% \sum和\int。设定上下界使用_和^
\[\sum_{i=1}^n i^2\]

\[\int_a^b x dx\]

综合示例:
\begin{align}       % 创建对齐的数学公式
    \hat{\bm{x}}^{(i)} 
    & \backsim \mathcal{M}_{\mathcal{D}}(\hat{\bm{x}}^{(i)} \mid \bm{x}^{(i)}) \label{eq:denoising_autoencoder_corrupt}   \\
    L_{DAE}(\theta,\phi)
    &=\frac{1}{n} \sum_{i=1}^{n}(\bm{x}^{(i)} - f_{\theta}(g_{\phi}(\hat{\bm{x}}^{(i)} ))) \label{eq:denoising_autoencoder_loss}
\end{align}


\section{常用数学结构}

矩阵:使用amsmath包的matrix环境,
可以创建不同种类的矩阵(如pmatrix,bmatrix等)。
这是一个带小括号的矩阵:
\[
\begin{pmatrix}
  1 & 2 & 3 \\
  4 & 5 & 6 \\
  7 & 8 & 9 \\
\end{pmatrix}
\]

方程组:使用amsmath包的align环境进行对齐的多行公式。
\begin{align}
    x + y &= z \\
    1 + 2 &= 3
\end{align}

